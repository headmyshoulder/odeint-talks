\section{Technical details}


\begin{frame}
  \tableofcontents[currentsection] 
\end{frame}

\subsection{Independent Algorithms}

\begin{frame}{Independent Algorithms}
\note{To achive maximum flexibility we wanted to implement the numerical algorithms independent from state type and computation details}

\begin{block}{Goal}
 Container- and computation-independent implementation of the numerical algorithms.  
\end{block}

\begin{block}{Benefit}
 High flexibility and applicability, \odeint\ can be used for virtually any formulation of an ODE.
\end{block}
 
\begin{block}{Approach}
 Detatch the algorithm from memory management and computation detail and make each part interchangeable.
\end{block}

\end{frame}

\begin{frame}{Required Computations}
 Typical mathematical computation to calculate the solution of an ODE ($\dot{\vec x} = \vec f(\vec x , t)$):

\begin{align*}
 \vec F_1 &= \vec f( \vec x_0 , t_0 ) \\
 \vec x' &= \vec x_0 + a_{21} \cdot \Delta t \cdot \vec F_1 \\
 \vec F_2 &= \vec f( \vec x' , t_0 + c_1\cdot\Delta t ) \\
 \vec x' &= \vec x_0 + a_{31} \cdot \Delta t \cdot \vec F_1 + a_{32} \cdot \Delta t \cdot \vec F_2 \\
         &\vdots \\
 \vec x_1 &= \vec x_0 + b_1\cdot \Delta t \cdot \vec F_1 + \dots + b_s\cdot \Delta t \cdot \vec F_s
\end{align*} 

\end{frame}

\begin{frame}[fragile]{Strucutural Requirements}
\begin{align*}
 {\color{red}\vec F_1} &= \vec f( {\color{red}\vec x_0} , {\color{blue}t_0} ) & %\\
 {\color{red}\vec x'} &= {\color{red}\vec x_0} + {\color{dark-green}a_{21}} \cdot {\color{blue}\Delta t} \cdot {\color{red}\vec F_1}
\end{align*}
 
Types:
\begin{itemize}
 \item {\color{red} vector type}, mostly, but not neccessarily, some container like \lstinline+vector<double>+
 \item {\color{blue} time type}, usually \lstinline+double+, but might be a multi-precision type
 \item {\color{dark-green} value type}, most likely the same as time type
\end{itemize}
\pause
\vspace{0.5em}

Function Call:
\begin{lstlisting}
void rhs( const vector_type &x , vector_type &dxdt , const time_type t )
{ /* user defined */ }

rhs( x0 , F1 , t ); //memory allocation for F1?
\end{lstlisting}


\end{frame}




\subsection{Memory Management}

\subsection{Computation Backend}

\subsection{Benefits}