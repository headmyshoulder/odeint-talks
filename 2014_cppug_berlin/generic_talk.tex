\documentclass{beamer}

\useoutertheme[footline=institute,subsection=false]{miniframes}
\usecolortheme{whale}
\setbeamercolor{titlelike}{parent=structure}
\useinnertheme{rounded}

\setbeamertemplate{footline}
{%
\begin{beamercolorbox}[wd=1.0\textwidth,ht=3ex,dp=1.5ex,left,leftskip=.5em]{title in head/foot}%
\usebeamerfont{title in head/foot}%
\hspace{1em}\insertshortauthor\hfill\insertshorttitle\hfill\raggedleft\insertframenumber{} of \inserttotalframenumber\hspace{1em}
\end{beamercolorbox}%
}

\usepackage{amsmath}
\usepackage{amsfonts}
\usepackage{amssymb}
%\usepackage{bbm}
\usepackage{units}

\usepackage[final]{listings}

\newcommand{\cmd}[1]{\mbox{\small{\texttt{#1}}}}
\newcommand{\e}{\mathrm{e}}
\newcommand{\op}[1]{\hat #1}
\newcommand{\rmd}{\mathrm{d}}
\newcommand{\mean}[1]{\langle #1 \rangle}
\newcommand\T{\rule{0pt}{2.6ex}}

\renewcommand{\emph}[1]{\textbf{\color{blue}#1}}

\title[{Metaprogramming Applied to Numerical Problems}]{\hbox{Metaprogramming Applied to Numerical Problems} \\ {\small A Generic Implementation of Runge-Kutta Algorithms}}

\author[Mario Mulansky]{Mario Mulansky\\ \small{Karsten Ahnert}}

\date{June 17, 2014} 

\institute{C++ Usergroup Berlin}

\titlegraphic{\includegraphics[height=2cm]{logo.pdf} \hfill
\includegraphics[width=4cm]{TU_Logo.pdf} \hfill
\includegraphics[height=2cm]{logo_mpipks.pdf}}

\beamertemplatenavigationsymbolsempty

\definecolor{dark-gray}{gray}{0.15}
\definecolor{light-gray}{gray}{0.8}
\definecolor{lighter-gray}{gray}{0.9}

\setbeamercolor{block title}{bg=light-gray} 
\setbeamercolor{block body}{bg=lighter-gray}

\definecolor{dark-green}{rgb}{0,0.4,0}
\definecolor{dark-red}{rgb}{0.2,0,0}

\lstset{
backgroundcolor=\color{lighter-gray},
frame=single,
basicstyle=\ttfamily\footnotesize,
tabsize=4,
keywordstyle=\color{dark-green},
identifierstyle=,
commentstyle=\color{dark-gray}\normalfont\rmfamily\itshape,
stringstyle=\color{dark-red},
language=c++,
showstringspaces=false
}


\begin{document}

\begin{frame}
\titlepage
\end{frame}

\section{Metaprogramming}

\begin{frame}{Metaprogramming}

\begin{block}{Wikipedia:}
\begin{quote}
 Metaprogramming is the writing of computer programs that write or manipulate other programs (or themselves) as their data, or that do part of the work at compile time that would otherwise be done at runtime.\\
\end{quote}
\end{block}

\pause
\begin{block}{In Short:}
 Metaprogramming means to write a program that writes a program that solves your problem.
\end{block}

\end{frame}

% \begin{frame}[fragile]{Example}
% 
% A bash script that writes a bash script that prints numbers from 1 to 992.
% 
%  \begin{lstlisting}[language=bash]
% #!/bin/bash
% # metaprogram
% echo "#!/bin/bash" > program
% for ((I=1; I<=992; I++)) do
%     echo "echo $I" >> program
% done
% chmod +x program  
%  \end{lstlisting}
% 
% 1000 lines of code generated in a minute.
% 
% However, not actually that useful.
% \end{frame}

\begin{frame}{Template Metaprogramming}
 \begin{itemize}
  \item Here, C++ Templates will be used for Metaprogramming
  \item This technique is called \textbf{Template Metaprogramming}
  \item It basically means using the template engine to generate a program from which the compiler creates then the binary
	\item C++ compilers always use the template engine (no additional compile step required)
 \end{itemize}

 \pause

 \begin{itemize}
	 \item Template engine and templates form a Turing-complete programming language
	 \item $\longrightarrow$ You can solve any problem with the template engine
 \end{itemize}

\end{frame}

\begin{frame}[fragile]{Example: Calculate factorial at compile time}
Compute the factorial of n:

\begin{lstlisting}{language=c++}
template< int n >
struct factorial {
  // recursive definition
  static const int value = factorial<n-1>::value*n;
};

template<>
struct factorial< 0 > {
  // condition to end recursion
  static const int value = 1;
};
\end{lstlisting}

\lstinline+factorial<5>::value+ gives $5!=120$, but computed at \textbf{compile time}!
\end{frame}

\begin{frame}{Why use Metaprogramming in numeric algorithms}
	
	\begin{itemize}
    \item Increase run time performance by shifting some calculations to compile time (but no floating point arithmetics!)
    \item Leads to more generic algorithms with no performance loss
	\end{itemize}
\pause
\vspace{1em}
\begin{Large}Downfalls:\end{Large}
\begin{itemize}
 \item Longer compile time
 \item Complicated compiler errors
\end{itemize}

\end{frame}


\begin{frame}[fragile]{A generic pow routine}
\begin{itemize}
 \item Compute $x^n$ where $n>0$ is known at compile time.
 \item Intelligent algorithm: $x^8 = ((x^2)^2)^2$, $\log_2 n + 1$ multiplications
 \item Implemented in \lstinline+boost/math/special_functions/pow.hpp+
\end{itemize}
\end{frame}

\begin{frame}[fragile]{A generic pow routine}
\begin{lstlisting}
template < int N , int M = N%2 >
struct pow_impl
{ // implementation for N even
  static double result( double base )
  { 
    double power = pow<N/2>::result( base );
    return power*power;
  }
};

template <int N>
struct pow_impl< N , 1 >
{ // specialization for N odd
  double operator()( double base )
  { 
    double power = pow<N/2>::result( base );
    return base*power*power;
  }
};
\end{lstlisting}
 
\end{frame}

\begin{frame}[fragile]{A generic pow routine}
\begin{lstlisting}
template <>
struct pow_impl< 1 , 1 >
{ // specialization to stop recursion
  double operator()( double base )
  { return base; }
};

template< int N >
double pow( double base ) { //convenience function
  return pow_impl<N>()::result( base ); 
}
\end{lstlisting}
Using \lstinline+pow<8>(x)+ the compiler sees \lstinline+t1=x*x; t2=t1*t1; return t2*t2;+
\end{frame}

\begin{frame}{Ordinary Differential Equations}
% ODEs are the typical way to describe physical, bioglogical, chemical, ... processes and thus play a fundamental role in mathematical modelling.
\vspace{.5em}
 \begin{minipage}{0.48\textwidth}
 \begin{center}
  Newtons equations

  \includegraphics[draft=false,width=0.8\textwidth]{solar_system.jpg}
 \end{center}
\end{minipage}
\pause
\begin{minipage}{0.48\textwidth}
 \begin{center}
  Reaction and relaxation equations (i.e. blood alcohol content, chemical reaction rates)
 \end{center}
\end{minipage}
\pause
\vspace{2ex}

\begin{minipage}{0.48\textwidth}
 \begin{center}
  Granular systems

  \includegraphics[draft=false,width=0.65\textwidth]{granular_system.png}
 \end{center}
\end{minipage}
\pause
\begin{minipage}{0.48\textwidth}
 \begin{center}
  Interacting neurons

  \includegraphics[draft=false,width=0.6\textwidth]{neuron.jpg}
 \end{center}
\end{minipage}
\pause
\vspace{2ex}

\begin{itemize}
 \item Many examples in physics, biology, chemistry, social sciences
 \item Fundamental in mathematical modelling
\end{itemize}

%  \begin{itemize}
%   \item Newton's equation of motion
%   \item Reaction-diffusion systems
%   \item Modelling of interacting neuronal networks
%  \end{itemize}
% 
% \pause
% \vspace{1em}
%  Also, ODEs are used as approximations to Partial Differential Equations for numerical treatments.

\end{frame}

\begin{frame}{Ordinary Differential Equations}

 A first order ODE is written in its most general form as:
 \begin{equation}
  \frac\rmd{\rmd t} \vec{x}(t) = \vec{f}(\vec x , t )
 \end{equation} 

 \begin{itemize}
  \item $\vec x(t)$ is the function in demand (here: trajectory)
  \item $t$ is the independent variable (here: time)
  \item $f(x ,t)$ is the rhs, governing the behavior of $x$
 \end{itemize}
 
Initial Value Problem (IVP):
 \begin{equation}
  \dot x = f( x , t ) ,\qquad x(t=0) = x_0
 \end{equation} 

\end{frame}

\begin{frame}{Examples}
 \begin{itemize}
  \item $\dot x = -\lambda x$ \hspace{1cm} solution: $x(t) = x_0 \e^{-\lambda t}$
  \item $ \ddot x = \omega^2 x \rightarrow \begin{cases}\dot x = p \\ \dot p = -\omega^2 x \end{cases}$ \hspace{0cm} solution: $x(t) = A \sin(\omega t + \varphi_0)$.

\pause

  \item Lorenz System: $
\begin{aligned}
 \dot x &= \sigma ( y - x ) \\
 \dot y &= x( R - z ) - y \\
 \dot z &= xy - \beta z.
\end{aligned} $ \hspace{0.5cm} solution: ?\\
 Chaotic system (for certain parameter values $\sigma, R , \beta$), hence the solution can not be written in analytic form.
 \end{itemize}

\pause 
\vspace{1em}
$\Longrightarrow$ numerical methods to solve ODEs are required for more complicated systems.
\end{frame}


\section{Generic Runge Kutta Scheme}

\begin{frame}{Runge-Kutta Scheme}

One class of algorithms to solve IVP of ODEs.

\begin{itemize}
 \item Discretized time $t \rightarrow t_n = t_0 + n\cdot h$ with (small) time step $h$
 \item Trajectory $x(t) \rightarrow x_n \approx x(t_n)$
 \item Iteration along trajectory: $x_0 \longrightarrow x_1 \longrightarrow x_2 \dots$
 \item One-step method: $x_1 = \Phi(x_0)$, $x_2 = \Phi(x_1)$, $\dots$ 
\end{itemize}

\pause
\begin{center}
  \includegraphics[width=0.8\linewidth]{x_discrete.pdf}
\end{center}

\end{frame}

\begin{frame}{Runge-Kutta Scheme}

Numerically solve the Initial Value Problem (IVP) of the ODE:
\begin{equation}
 \dot x(t) = f(x,t), \qquad x(t=0) = x_0.
\end{equation}
A Runge-Kutta scheme with $s$ stages and parameters $c_1 \dots c_s$, \hspace{.5em}$a_{21}, a_{31} , a_{32} , \dots , a_{s s-1}$ and $b_1 \dots b_s$
gives the approximate solution for $x_1 \approx x(h)$ starting at $x_0$ by computing:
\begin{equation}
 x_1 = x_0 + h\sum_{i=1}^s b_i F_i \qquad \text{where} \qquad F_i = f( x_0 + h\sum_{j=1}^{i-1} a_{ij} F_j , h c_i ).
\end{equation}

This approximate solution $x_1$ is exact up to some order $p$.

Repeating the whole procedure brings you from $x_1$ to $x_2$, then to $x_3$ and so on.

\end{frame}

\begin{frame}
 At each stage $i$ the following calculations have to be performed ($y_1 = x_0$) :
\begin{align*} \label{eqn:rk_scheme}
 F_i &= f( y_{i} , h c_i ), \qquad y_{i+1} = x_0 + h\sum_{j=1}^{i} a_{i+1,j} F_j, \qquad i=1\dots s-1\quad 
\\
 F_s &= f(y_{s} , h c_s ) , \qquad x_1 = x_0 + h\sum_{j=1}^{s} b_{j} F_j.
\end{align*}
The parameters $a$, $b$ and $c$ define the so-called Butcher tableau.

\end{frame}


\begin{frame}{Butcher Tableau}
Parameters $a$, $b$, and $c$ are typically written as Butcher tableau:

\begin{center}
 \begin{tabular}{c|ccccc}
   $c_1$ &  & & & & \\
   $c_2$ & $a_{2,1}$ & & & & \\
   $c_3$ & $a_{3,1}$ & $a_{3,2}$ & & & \\
   $\vdots$ & $\vdots$ &  & $\ddots$ & & \\
   $c_s$ & $a_{s,1}$ & $a_{s,2}$ & $\dots$ & $c_{s,s-1}$ & \\
  \hline 
    & $b_1$ & $b_2$ & $\dots$ & $b_{s-1}$ & $b_s$ \\
 \end{tabular}
\end{center}
 
The Butcher Tableau fully defines the Runge-Kutta scheme.

Each line of the tableau represents one stage of the scheme.
\end{frame}

\begin{frame}{Generic Runge-Kutta Algorithm}
 
\begin{Large}Idea:\end{Large}
  \begin{itemize}
    \item Write a Metaprogram that creates Runge-Kutta algorithms
    \item Metaprogram input: Parameters of the RK scheme (Butcher Tableau)
    \item Main objective: \textbf{Resulting program should be as fast as direct implementation}
  \end{itemize}

\vspace{1em}
With such a Metaprogram you can implement any new Runge-Kutta scheme by just providing the Butcher tableau.

\begin{itemize}
 \item Decrease in programming time
 \item Less bugs
 \item Better maintainability
\end{itemize}

\end{frame}


\begin{frame}
 \includegraphics[width=\textwidth]{fig/generic_rk_diagram.pdf}
\end{frame}

\section{Implementation}

\begin{frame}[fragile]{Explicit Non-Generic Implementation}
Given parameters \lstinline+c_i , a_ij , b_i+ 
\begin{lstlisting}
F_1 = f( x , t + c_1*dt );
x_tmp = x + dt*a_21 * F_1;

F_2 = f( x_tmp , t + c_2*dt );
x_tmp = x + dt*a_31 * F_1 + dt*a_32 * F_2;

// ...

F_s = f( x_tmp , t + c_s*dt );
x_end = x + dt*b_1 * F_1 + dt*b_2 * F_2 + ... 
          + dt*b_s * F_s; 
\end{lstlisting}

Not generic: Each stage written hard coded -- you have to adjust the algorithm when implementing a new scheme.
\end{frame}


\begin{frame}[fragile]{Run Time Implementation}
Given parameters \lstinline+a[][] , b[] , c[]+.
\begin{lstlisting}
F[0] = f( x , t + c[0]*dt );
x_tmp = x + dt*a[0][0] * F[0];

for( int i=1 ; i<s-1 ; ++i )
{
  F[i] = f( x_tmp , t + c[i]*dt );
  x_tmp = x;
  for( int j=0 ; j<i+1 : ++j )
    x_tmp += dt*a[i][j] * F[j];
}

F[s-1] = f( x_tmp , t + c[s-1]*dt );
x_end = x;
for( int j=0 ; j<s : ++j )
  x_end += dt*b[j] * F[j];
\end{lstlisting}

\pause
\textbf{Generic, but factor 2 slower than explicit implementation!}
\end{frame}

\begin{frame}[fragile]{Why Bad Performance}

The run time generic code is hard to optimize for the compiler, because:

\begin{itemize}
 \item Double \lstinline+for+ loop with inner bound depending on outer loop variable.
 \item 2D array \lstinline+double** a+ must be dynamically allocated:
\begin{lstlisting}
a = new double*[s];
for( int i=0 ; i<s ; ++i )
  a[i] = new double[i+1];
a[0][0] = ...; 
a[1][0] = ...; a[1][1] = ...; 
...
\end{lstlisting}
$\longrightarrow$ lives on heap, harder to be optimized compared to stack.
 \item Many more issues possible (optimizers are rather complex).

\end{itemize}

\end{frame}


\begin{frame}{What to do?}

\begin{block}{Idea:}
 Use template engine to generate code that can be efficiently optimized by the Compiler.
\end{block}

\vspace{1em}
\pause
More specifically, we will use Template Metaprogramming to:
\begin{itemize}
 \item Generate fixed size arrays: \lstinline+a_1[1] , a_2[2] , ... , a_s[s]+
 \item Unroll the outer \lstinline+for+-loop (over stages \lstinline+s+) so the compiler sees sequential code.
\end{itemize}

As result, the code seen by the compiler/optimizer (after resolving templates) is very close to the non-generic version and thus as fast, hopefully.
\end{frame}

\begin{frame}[fragile]{The Generic Implementation}

Define a structure representing one stage of the Runge-Kutta scheme:

\begin{lstlisting}
template< int i >
struct stage // general (intermediate) stage, i > 0
{
  double c; // parameter c_i
  array<double,i> a; // parameters a_i+1,i ... a_i,i
                     // b_1 .. b_j for the last stage
};
\end{lstlisting}

Given an instance of this stage with \lstinline+c+ and \lstinline+a+ set appropriately the corresponding Runge-Kutta stage can be calculated. 

%\vspace{1em}
%\begin{footnotesize}Note: \lstinline+array<double,i>+ is the C++ equivalent of \lstinline+double[i]+.\end{footnotesize}
\end{frame}


\begin{frame}[fragile]{The Generic Implementation}
%Perform the calculation for this stage.
\begin{lstlisting}
// x , x_tmp , t , dt and F defined outside
template< int i >
void calc_stage( const stage< i > &stage )
{  // performs the calculation of the i-th stage
  if( i == 1 ) // first stage?
    F[i-1] = f( x , t + stage.c * dt );
  else
    F[i-1] = f( x_tmp , t + stage.c * dt );

  if( i < s ) { // intermediate stage?
    x_tmp = x;
    for( int j=0 ; j<i : ++j )
      x_tmp += dt*stage.a[j] * F[j];
  } else {   // last stage
    x_end = x;
    for( int j=0 ; j<i : ++j )
      x_end += dt*stage.a[j] * F[j];
  }
}
\end{lstlisting}
\end{frame}

\begin{frame}[fragile]{The Generic Implementation}
Generate list of stage types: \lstinline+stage<1> , stage<2>, ... , stage<s>+ using Boost.MPL (MetaProgramming Library) and Boost.Fusion.

\begin{lstlisting}[basicstyle=\ttfamily\tiny]
typedef mpl::range_c< int , 1 , s > stage_indices;

typedef typename fusion::result_of::as_vector
< typename mpl::push_back
  < typename mpl::copy
    < stage_indices,
      mpl::inserter
      <
        mpl::vector0<> ,
        mpl::push_back< mpl::_1 , stage_wrapper< mpl::_2 , stage > >
      >
    >::type , stage< double , stage_count , last_stage >
  >::type
>::type stage_vector_base; //fusion::vector< stage<1> , stage<2> , ... , stage<s>

struct stage_vector : stage_vector_base
{
  // initializer methods
  stage_vector( const a_type &a , const b_type &b , const c_type &c )
  {
    // ...
  }
}
\end{lstlisting}

\end{frame}

\begin{frame}[fragile]{The Generic Implementation}
Parameter types for \lstinline+a+, \lstinline+b+ and \lstinline+c+:
\begin{lstlisting}[basicstyle=\ttfamily\tiny]
typedef typename fusion::result_of::as_vector
< typename mpl::copy
  < stage_indices ,
    mpl::inserter
    < mpl::vector0< > ,
      mpl::push_back< mpl::_1 , 
                      array_wrapper< double , mpl::_2 > >
    >
  >::type
>::type a_type; //fusion::vector< array<double,1> , array<double,2> , ... >

typedef array< double , s > b_type;
typedef array< double , s > c_type;
\end{lstlisting}

\pause
Instead of a dynamically allocated \lstinline+double**+ the compiler/optimzier sees fixed size arrays: \lstinline+array<double,1>+ , \lstinline+array<double,2>+, ... \\
$\longrightarrow$ \textbf{better optimization possibilities}
\end{frame}

\begin{frame}[fragile]{The Generic Implementation}
The actual Runge-Kutta step (details ommited):
\begin{lstlisting}[basicstyle=\ttfamily\tiny]
fusion::for_each( stages , 
                  calc_stage_caller( f , x , x_tmp , x_end , F , t , dt ) );
\end{lstlisting}
Remember: \lstinline+stages+ is \lstinline+fusion::vector< stage<1> , stage<2> , ... >+
For each of the \lstinline+stages+, \lstinline+calc_stage+ gets called, but the \lstinline+for_each+-loop is \textbf{executed by the compiler!}

\pause
\vspace{1em}
The compiler/optimizer sees sequential code:
\begin{lstlisting}
calc_stage( stage_1 ); // stage_1 is an 
calc_stage( stage_2 ); // instance of stage<1>
...                    // similar for stage_2 ...
calc_stage( stage_s );
\end{lstlisting}

$\longrightarrow$ \textbf{better optimization possibilities}

\end{frame}


\begin{frame}[fragile]{The Generic Stepper}
Provide some handy interface to the generic algorithm:
 \begin{lstlisting}[basicstyle=\ttfamily\tiny]
template< int s >
class generic_runge_kutta
{
public:
  generic_runge_kutta( const coef_a_type &a ,
                       const coef_b_type &b ,
                       const coef_c_type &c )
    : m_stages( a , b , c )
  { }

  void do_step( System f , const state_type &x , const double t , 
                state_type &x_out , const double dt )
  {
    fusion::for_each( m_stages , calc_stage_caller( f , x , m_x_tmp , x_out , 
                                                    m_F , t , dt ) );
  }

private:
  stage_vector m_stages;
  state_type m_x_tmp;

protected:
  state_type m_F[s];
};
\end{lstlisting}

\end{frame}


\section{Performance}

\begin{frame}[fragile]{Example: Runge-Kutta 4}

\begin{center}
Butcher Tableau: 
 \begin{tabular}{c|cccc}
   0 &  & &  & \\
   0.5 & 0.5 & & & \\
   0.5 & 0 & 0.5 & & \\
   1.0 & 0 & 0 & 1.0 & \\
  \hline 
    & 1/6 & 1/3 & 1/3 & 1/6 \\
 \end{tabular}
\end{center}


\begin{lstlisting}[basicstyle=\ttfamily\scriptsize]
 // define the butcher array
const array< double , 1 > a1 = {{ 0.5 }};
const array< double , 2 > a2 = {{ 0.0 , 0.5 }};
const array< double , 3 > a3 = {{ 0.0 , 0.0 , 1.0 }};

const a_type a = fusion::make_vector( a1 , a2 , a3 );
const b_type b = {{ 1.0/6.0 , 1.0/3.0 , 1.0/3.0 , 1.0/6.0 }};
const c_type c = {{ 0.0 , 0.5 , 0.5 , 1.0 }};

// create the stages with the rk4 parameters a,b,c
generic_runge_kutta< 4 > rk4( a , b , c );
// do one rk4 step
rk4.do_step( lorenz , x , 0.0 , x , 0.1 ); 
\end{lstlisting}
\end{frame}


\begin{frame}{Performance}
Did we achieve our aim? Test RK4 on Lorenz System!

\pause
%\vspace{1em}
\begin{columns}
\begin{column}{0.55\linewidth}
\includegraphics[width=1.0\linewidth]{fig/perf_rk4.png}
\end{column}
\begin{column}{0.4\linewidth}
\begin{scriptsize}
  \begin{tabular}[b]{c}
    \textbf{Processors:} \\
    Intel Core i7 830 \\
    Intel Core i7 930 \\
    Intel Xeon X5650 \\
    Intel Core2Quad Q9550 \\
    AMD Opteron 2224 \\
    AMD PhenomII X4 945 \\
    \hline
    \textbf{Compilers:} \\
    gcc 4.3 , 4.4 , 4.5 , 4.6 \\
    intel icc 11.1 , 12.0 \\
    msvc 9.0
  \end{tabular}               
\end{scriptsize}
\end{column}
\end{columns}

\pause
\begin{center}
\textbf{Yes!}
\end{center}
\pause%
\begin{small}
\begin{itemize}
\item On modern compilers (Intel 12, gcc 4.5/4.6) as fast as explicit code.
\item Older compilers might produce slightly worse performant code.
\item Always factor 2 better than run time generic implementation.
\end{itemize}
\end{small}

\end{frame}

\begin{frame}{Conclusions}
We implemented a generic Runge-Kutta algorithm that executes \textbf{any} RK scheme and has the following properties:
\begin{itemize}
 \item Parameters (Butcher Tableau) can be defined in a natural way as C++ Arrays
 \item By virtue of Template Metaprogramming our code is as fast as direct implementation of the specific scheme
 \item \textbf{Major improvement (factor 2) compared to generic run time implementation} (but some increase in compile time)
 \item Embedded methods with error estimate can also be easily covered in a generic way
 \item This technique can be applied to other numerical problems, e.g.\ spline fitting, ...
\end{itemize}

\end{frame}

\begin{frame}

\begin{center}
\includegraphics[width=0.8\textwidth]{../odeint_logo.pdf}
\end{center}
 
\end{frame}



\end{document}